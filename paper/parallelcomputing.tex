\documentclass{article}[11pt,subeqn]

\usepackage{fancyhdr}
\pagestyle{fancy}
\usepackage[pdftex]{graphicx}
\usepackage{setspace}
\usepackage{framed}
%\usepackage{lastpage}
\usepackage{amsmath}
\usepackage[utf8]{inputenc}

\title{We Just Ran Two Million Regressions Faster}
\author{Damian C. Clarke \and George Vega Yon}


\setlength\topmargin{-0.375in}
\setlength\headheight{20pt}
\setlength\headwidth{5.8in}
\setlength\textheight{8.8in}
\setlength\textwidth{5.8in}
\setlength\oddsidemargin{0.4in}
\setlength\evensidemargin{-0.5in}
\setlength\parindent{0.25in}
\setlength\parskip{0.25in}

\usepackage{natbib}
\bibliographystyle{abbrvnat}
\bibpunct{(}{)}{;}{a}{,}{,}

\usepackage{lscape}
\usepackage{rotating}
%\usepackage{multirow}
\usepackage{rotating,capt-of}
\usepackage{array}

\usepackage[update,prepend]{epstopdf}

\usepackage[font=sc]{caption}

%NEW COMMANDS
\newcommand{\Lagr}{\mathcal{L}}
\newcommand{\vect}[1]{\mathbf{#1}}
\newcolumntype{P}[1]{>{\raggedright}p{#1\linewidth}}

%\usepackage{appendix}
\usepackage{booktabs}
%\usepackage{cleveref}

\fancyhead{}
\fancyfoot{}
\fancyhead[L]{\textsc{Parallel Computing}}
\fancyhead[R]{\textsc{Working Paper}}
\fancyfoot[C]{\textsc{\thepage}} %\ of  \pageref{LastPage}
\fancyfoot[R]{DCC \& GVY}

\begin{document}
\begin{spacing}{1.25}

\maketitle

\begin{abstract}

\end{abstract}

\section{Introduction}
The multilinear regression is the workhorse tool used in a large majority of applied economic work.  Regression analysis is both a powerful and---as the 
title of \citet{SalaiMartin1997} suggests---ubiquitous method in economics.  However, despite great increases in the efficiency of modern computation and
considerable advances in the complexity and availability of data, the basic way economists \emph{physically} run regressions has changed little over
the past decade.  Fundamentally economists rely on the inbuilt ability of statistical programs such as Stata, MATLAB, R and so forth, to project vectors
of observations onto each other, and hence estimate the coefficients of interest from their underlying equation.

This methodology, whilst completely effective in doing its job, is relatively inefficient when considering both computational and human resources.  The
delay in calculation time experienced when using large datasets is often considerable, meaning that researchers or their research assistants are forced
to ``run regressions'' and then wait some non-trivial time to study regression outputs.  Such a process may imply interruption to the economist's ability 
to undertake research, and at the very least implies tying up the program or computer (or terminal's) analytical capacity whilst the regression ``runs''.  

In this paper we provide a methodology by which the efficiency of multilinear regressions -- and indeed a large range of economic calculations -- can be
increased by the order of up to XX.  Whilst this methodology is not novel in the physical sciences, corporate enterprises, or even in some iterative economic
processes (see for example \citeauthor{aldrich2011}, \citeyear{aldrich2011}), we believe that this paper represents its first analysis in terms of linear 
regression in which its successful(??) application is demonstrated in potentially the most widely used statistical program in applied economics; Stata.  

By taking advantage of parallel computing and \ldots


\section{Why Parallel computing?}

\begin{itemize}
\item Huge data available (specially administrativa data) allow researchers and governments to replicate and inquire deeply in reality. Big data-ware houses are been built every where motive my open-data and transparency movements.
\item Impressive advances in GPU technologies taking computation into a whole new level through parallel computation. In less than ten years, the computational capabilities of a home computer have multiplied thousands of times\footnote{The new NVIDIA video card for laptos, GeForce GTX 680M, has 1344 cores.} 
\item Thus, the necessity of making assumptions such as ceteris-paribus is no longer necessary. Researchers can now simulate complete societies from the comfort of their desks just using mid capacity computers.
\end{itemize}

\section{The Multilinear Regression and Parallel Computing}
The nature of the matrix algebra employed in multilinear regression means that efficiency gains associated with parallel computing are particularly 
large relative to traditional single-core computes.  As each variable is expressed as a vector and each of these vectors must be projected onto itself 
and each other vector (variable), this implies that for each pair of variables $(p,q)$ a stepwise sum must be performed of each observation's coproduct 
$x_{pi}x_{qi}$, for $i\in \{1,\ldots,n\}$.  In this way, as the number of observations $n$ in a set of data increases, the quantity of individual 
multiplications must increase in line with this for each pair of variables.

Hence, rather than working stepwise through a summation of $n$ units,\footnote{This is to say first multiplying $x_1$ by $x_1$, then $x_2$
by $x_2$ and so forth up to $x_n$ by $x_n$, whereby each $x$ is a scalar and the subindex represents observation number.} parallel computing allows 
us to split each vector and resulting summation into $c$ sub-vectors of $\frac{n}{c}$ observations, where $c$ represents the number of cores which 
the computer contains in its CPU or GPU:
\begin{equation}
\label{eqn:parallelised}
\sum_{c=1}^C\left(\sum_{i\forall i\in c}^{\frac{n}{c}}x^2_i\right)_c.
\end{equation}
Furthermore, given that the process of multiplication and summation suffers from virtually no increasing returns to scale where subsequent
additions to the vector size imply decreasing marginal additions to calculation time, as $c$ increases in number we expect that the amount of time that
the regression takes to ``run'' should decrease $c$-fold. 

Formally, each multilinear regression, expressed in terms of a matrix $\vect{X}$ of independent variables and a vector $\vect{y}$ representing the 
dependent variable, requires two computationally expensive matricial calculations: $\vect{X'X}$ and $\vect{X'y}$, and the subsequent (faster) combination
of the two resulting matrices to produce the vector of estimands. Given that each individual variable in $\vect{X}$ must be multiplied by itself and
each other variable in $\vect{X}$ (and likewise for the variable $y$), the summation (\ref{eqn:parallelised}) is performed for each variable pair as
per (\ref{eqn:OLSMat_b}):
\begin{subequations}
\begin{eqnarray}
\label{eqn:OLSMat}
\vect{(X'X)}&=& %^{-1}(X'y)
\begin{bmatrix}
x_{11} &  x_{12} & \ldots & x_{1n}\\
x_{21} &  x_{22} & \ldots & x_{2n}\\
\vdots & \vdots  & \ddots &  \vdots\\
x_{k1} &  x_{k2} & \ldots & x_{kn}\\
\end{bmatrix}%_{k\times n}
\begin{bmatrix}
x_{11} &  x_{21} & \ldots & x_{k1}\\
x_{12} &  x_{22} & \ldots & x_{k2}\\
\vdots & \vdots  & \ddots &  \vdots\\
x_{1n} &  x_{2n} & \ldots & x_{kn}\\
\end{bmatrix}\\ \label{eqn:OLSMat_a}
&\equiv&
\begin{bmatrix}
\sum^n_{i=1}x^2_{1i} &  \sum^n_{i=1}x_{1i}x_{2i} & \ldots & \sum^n_{i=1}x_{1i}x_{ki}\\
\sum^n_{i=1}x_{2i}x_{1i} &  \sum^n_{i=1}x^2_{2i} & \ldots & \sum^n_{i=1}x_{2i}x_{ki}\\
\vdots & \vdots  & \ddots &  \vdots\\
\sum^n_{i=1}x_{ki}x_{1i} &  \sum^n_{i=1}x_{ki}x_{2i} & \ldots & \sum^n_{i=1}x^2_{ki}\\
\end{bmatrix}. \label{eqn:OLSMat_b}
\end{eqnarray}
\end{subequations}

\nocite{vespignani2011modelling}
\newpage
\bibliography{bibliography.bib}

\end{spacing}
\end{document}